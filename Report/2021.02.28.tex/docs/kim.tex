\section{Kim(1987)}
Over the past period of time, we have read Kim(1987) in detail and learned some things that need attention.We Know the unsteady Navier-Stokes equations are solved numerically at a Reynolds number of 3300, based on thc mean centreline velocity and channel half-width, with about $4*10^4$ grid points (192 * 129 * 160 in x, y, z).By reading this paper, we learned that the following parameters need to be paid attention to.
\subsection{Mean properties}
This section mainly discussed the mean-velocity profile.The skin-friction coefficient, bulk  mean  velocity, displacement and momentum thicknesses are also computed from the computed mean-velocity profile. 
\subsection{Turbulence intensities}
This section mainly discussed the turbulence intensities near the wall normalized by the local mean velocity and root-mean-square pressure fluctuations normalized by the wall shear velocity.We need notice $u_{rms}$,$v_{rms}$,$w_{rms}$ and $p_{rms}$.
\subsection{Reynolds shear stress}
This section mainly discussed the Reynolds shear stress normalized by the wall shear velocity and correlation coefficient of u' and u'.We need notice $\overline{u'v'}$.
\subsection{Near-wall behaviour of Reynolds stresses}
This section mainly discussed the near-wall behaviour of Reynolds stresses We need notice $u_{rms}^+$,$v_{rms}^+$ and $w_{rms}^+$.
\subsection{Vorticity}
This section mainly discussed the comparison of the near-wall behaviour of the normal velocity fluctuation.We need notice root-mean-square vorticity fluctuations normalized by the mean shear.($\omega_x \upsilon/u_\tau^2$,$\omega_x \upsilon/u_\tau^2$ and $\omega_x \upsilon/u_\tau^2$).
\subsection{Quadrant analysis}
This section mainly discussed the analysis divides the Reynolds shear stress into four categories according to the signs of u’ and v’. 
\subsection{Higher-order statistics}
This section mainly discussed that agreements among the computed and measured  values are satisfactory for u' and w', but there exists a significant discrepancy for v', especially for the flatness factor in the vicinity of the wall.
\subsection{Summary}
In particular, the computed Reynolds stresses - both the normal and the shear stresses - are consistently lower than the measured values, while the computed vorticity  fluctuations at the wall are higher than  the experimental values. 
One source of the discrepancy might be related to the measurement of the wall-shear velocity $u_\tau$.
Another source of the discrepancy may be the test section of the oil channel used in the aforementioned experiments. 
